\documentclass{article}
\usepackage[top=2.5cm]{geometry}
\usepackage{amsmath}
\title{HW1 Classical Electromagnetism, Fall 2016}
\author{Deheng Song}
\date{\ }
\begin{document}
  \maketitle
  %--------------1.1--------------------%
  \section*{1.1}
  (a). The electric field inside a conductor is zero when the conductor is under the electrostatic equilibrium. Use the Gauss's law on the
  conductor:
  \[ \nabla\cdot\vec E =\frac{\rho}{\varepsilon}, \]
  we find:
  \[ \rho=0\ inside\ the \ conductor\]
  since the electric field $\vec E$ vanished everywhere. The excess charges placed on a conductor can only lie on the surface.\par
  $~$\\
  (b). For a closed path, the curl integral
  \[ \oint\vec E\cdot d\vec l=0. \]
  No matter what a charge distribution outside the conductor is, we can find a curl part in the conductor and part in the hollow, and
  \[\int_{cond} \vec E\cdot d\vec l+\int_{hol} \vec E\cdot d\vec l=0 \]
  The first part of the integral will always be zero since there is no electric field in the conductor. For the arbitrariness of the
  curl, the integral could be zero only when $\vec E=0$ everywhere inside the hollow. So the conductor shields the hollow from fields due to charges
  outside.\par
  For the charges placed inside the hollow, it's easy to find a Gussian surface in the conductor surrounding the hollow. The electric
  field and charge in the conductor are zero, and
  \[\oint \vec{E}\cdot\vec{n}da=0. \]
  So the total charges inside the Gussian surface is zero. If the charges placed in the hollow is $Q$, there must be $-Q$ induced on the
  interior surface of the conductor. To reach the electrostatic equilibrium, the charge induced on the exterior surface will be $Q$ too.
  For a Gussian surface surrounding the whole conductor,
  \[ \oint\vec E\cdot\vec nda=\frac{Q}{\varepsilon_0} \]
  which means that the conductor does not shield its exterior from the fields due to charges inside it.
  \par
  $~$\\
  (c). The Gauss's law tells us that at the surface of the conductor, the normal component of electric fields:
  \[(\vec{E}_2-\vec{E}_1)\cdot \vec n=\sigma/\varepsilon_0 \]
  Since the electric field on the side 1 (inside) is zero, we find:
  \[\vec{E}_2\cdot n =E_{2n}=\sigma/\varepsilon_0 \]
  And the parallel component of $\vec{E}$ is continous:
  \[ (\vec{E}_2-\vec{E}_1)\cdot\vec{t}=0\]
  Also we find,
  \[ E_{2t}=E_{1t}=0.\]
  Thus the electric filed is normal to the surface of the conductor and the magnitude is $\sigma/\varepsilon_0$.
  \pagebreak
  %-----------------1.5-------------------------%
  \section*{1.5}
  \begin{align*}
    \nabla^2\Phi&=\nabla^2[\frac{q}{4\pi\varepsilon_0}\frac{e^{-\alpha r}}{r}(1+\frac{\alpha r}{2})]\\
    &=\frac{q}{4\pi\varepsilon_0}\nabla^2(\frac{e^{-\alpha r}}{r}+\frac{\alpha}{2}e^{-\alpha r})
  \end{align*}
  The Laplacian in the spherical coordinates can be writen as:
  \[ \nabla^2=\frac{1}{r^2}\frac{\partial}{\partial r}(r^2\frac{\partial}{\partial r})+\frac{1}{r^2\sin^2\phi}\frac{\partial^2}{\partial \theta}
  +\frac{1}{r^2\sin\phi}\frac{\partial}{\partial\phi}(\frac{1}{\sin\phi}\frac{\partial}{\partial\phi})
  \]
  When $r>0$,
  \begin{align*}
      \nabla^2\Phi&=\frac{q}{4\pi\varepsilon_0}\frac{1}{r^2}\frac{\partial}{\partial r}r^2\frac{\partial}{\partial r}(\frac{e^{-\alpha r}}{r}
      +\frac{\alpha}{2}e^{-\alpha r})\\
      &=\frac{q}{4\pi\varepsilon_0}\frac{1}{r^2}\frac{\partial}{\partial r}(-e^{-\alpha r}-\alpha re^{-\alpha r}-\frac{\alpha^2 r^2}{2}e^{-\alpha r})\\
      &=\frac{q}{4\pi\varepsilon_0}\frac{1}{r^2}(\alpha e^{-\alpha r}+\alpha^2 re^{-\alpha r}-\alpha e^{-\alpha r}-\alpha^2 re^{-\alpha r}+\frac{\alpha^3 r^3}{2}e^{-\alpha r})\\ 
      &=\frac{q\alpha^3}{8\pi\varepsilon_0}e^{-\alpha r}
  \end{align*}
When $r\rightarrow 0$,
\begin{align*}
    \nabla^2\Phi&=\frac{q}{4\pi\varepsilon_0}\nabla^2(\frac{1}{r}+\frac{\alpha}{2})\\
    &=\frac{q}{4\pi\varepsilon_0}(\nabla^2\frac{1}{r}+\nabla^2\frac{\alpha}{2})
\end{align*}
The last term vanish and $\nabla^2 \frac{1}{r}=-4\pi\delta(r)$,
\begin{align*}
    \nabla^2\Phi&=-\frac{q}{\varepsilon_0}\delta(r)
\end{align*}
Add the two part, and $\nabla^2\Phi=-\frac{\rho}{\varepsilon_0}$,
\begin{align*}
    \rho=q\delta(r)-\frac{q\alpha^3}{8\pi}e^{-\alpha r}
\end{align*}
The first term $q\delta(r)$ indicates a point charge $q$ lies on the orgin, which can be interpret as the charge on the hydrogen nucleus.
And the last term describes the charge distribution around the nucleus.
\begin{align*}
  -\frac{q\alpha^3}{8\pi}\int_0^{2\pi}\int_0^\pi\int_0^\infty e^{-\alpha r}r^2\sin\theta drd\phi d\theta=-q
\end{align*}
remains the total charge of the hydrogen atom neutral.
\pagebreak
%-------------------1.10---------------%
\section*{1.10}
The mathematical form of the mean value theorem can be written as
\[ \Phi(\vec x)=\frac{\oint_S\Phi(\vec x')da'}{4\pi R^2} \]
where $\Phi(\vec x')$ is the potential on the surface of a sphere centered on $\vec x$ and $R$ is the radius of the sphere. This motivates us
to consider the Green's theorem on Dirichlet boundary:
\[ \Phi(\vec x)=\frac{1}{4\pi\varepsilon_0}\int_V\rho(\vec x')G_D(\vec x,\vec x')d^3d-\frac{1}{4\pi}\oint_S\Phi(\vec x')\frac{\partial G_D}
{\partial n'}da' \]
For charge free space $\rho=0$,
\begin{align*}
\Phi(\vec x)&=-\frac{1}{4\pi}\oint_S\Phi(\vec x')\frac{\partial G_D}{\partial n'}da\\
            &=\frac{\oint_S\Phi(\vec x')da'}{-4\pi\frac{1}{\partial G_D/\partial n'}}
\end{align*}
\[ G_D(\vec x,\vec x')=\frac{1}{|\vec x-\vec x'|}+F(\vec x,\vec x') \]
vanish on the surface. Since $1/|\vec x-\vec x'|=1/R$ on the surface, it's easy to find $F=-1/R$,
\[ G_D(\vec x,\vec x')=\frac{1}{|\vec x-\vec x'|}-\frac{1}{R} \]
On the surface of the sphere,
\begin{align*}
\frac{\partial G_D}{\partial n'}&=\frac{\partial (1/|\vec x-\vec x'|-1/R)}{\partial n'}\\
&=\frac{\partial (1/|\vec x-\vec x'|)}{\partial n'}-\frac{\partial (1/R)}{\partial n'}\\
&=-\frac{1}{R^2}
\end{align*}
Thus,
  \begin{align*}
  \Phi(\vec x)&=-\frac{1}{4\pi}\oint_S\Phi(\vec x')\frac{\partial G_D}{\partial n'}da\\
              &=\frac{\oint_S\Phi(\vec x')da'}{-4\pi\frac{1}{\partial G_D/\partial n'}}\\
              &=\frac{\oint_S\Phi(\vec x')da'}{4\pi R^2}
  \end{align*}
The mean value theorem is proved.
\pagebreak
%------------------1.12----------------%
\section*{1.12}
Let $\phi=\Phi$ and $\psi=\Phi'$, the Green's theorem tells us that:
\[ \int_V(\Phi'\nabla^2\Phi-\Phi\nabla^2\Phi')d^3x=\oint_S[\Phi'\frac{\partial\Phi}{\partial n}-\Phi\frac{\partial\Phi'}{\partial n}]da\]
We know from the Possion's equation that
\[ \nabla^2\Phi=\frac{\rho}{\varepsilon_0}\qquad\nabla^2\Phi'=\frac{\rho'}{\varepsilon_0} \]
Since the surface is conducting, we find the normal component of the elelctric field on the surface
\[ E_n=-\frac{\sigma}{\varepsilon_0} \]
With $E_n=-\frac{\partial\Phi}{\partial n}$, we get
\[\frac{\partial\Phi}{\partial n}=\frac{\sigma}{\varepsilon_0}\qquad\frac{\partial\Phi'}{\partial n}=\frac{\sigma'}{\varepsilon_0} \]
Substitute into the Green's theorem,
\[\int_V(\Phi'\frac{\rho}{\varepsilon_0}+\Phi\frac{\rho'}{\varepsilon_0})d^3x=\oint_S[\Phi'\frac{\sigma}{\varepsilon_0}-\Phi\frac{\sigma'}{\varepsilon_0}]da \]
Thus,
\[ \int_V\rho'\Phi d^3x+\oint_S\sigma'\Phi da=\int_V\rho\Phi' d^3x\oint_S\sigma\Phi' da \]
\end{document}
