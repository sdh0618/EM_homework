\documentclass{article}
\usepackage{amsmath}
\usepackage{amsfonts}
\usepackage{latexsym}
\usepackage{graphicx}
\newcommand{\comment}[1]{}
\newcommand{\field}[1]{\mathbb{#1}} % requires amsfonts
\newcommand{\pd}[2]{\frac{\partial#1}{\partial#2}}
\begin{document}
\title{HW9, Classical Electromagnetism}
\author{Deheng Song}
\maketitle
\section*{Jackson 5.20(a)}

From equation (5.12),
\[ \vec F=\int \vec J(\vec x)\times\vec B(\vec x)d^3x \]
In the region the body occupies,
\begin{align*}
  \vec J_M=\nabla\times\vec M\\
  \vec K_M =\vec M\times\vec n
\end{align*}
Outside the body, in absence of the macroscopic conduction currents,$ \vec J=0$.
This yields that,
\begin{align*}
  \vec F=\int_V(\nabla\times\vec M)\times\vec Bd^3x+\int_S(\vec M\times \vec n)\times B\cdot d\vec a 
\end{align*}
For the first term,
\begin{align*}
  (\nabla\times\vec M)\times\vec B&=-\vec B\times(\nabla\times\vec M)\\
                                  &=(\vec M\cdot\nabla)\vec B+(\vec B\cdot\nabla)\vec M+\vec M\times(\nabla\times\vec B)-\nabla(\vec M\cdot\vec B)
\end{align*}
Since $\vec J=0$, we have $\vec\times B=\vec J=0$.\\
For the second terms,
\begin{align*}
  (\vec M\times\vec n)\times\vec B=-\vec B\times(\vec M\times\vec n) \\
  =\vec n(\vec B\cdot\vec M)-\vec M(\vec B\cdot\vec n)
\end{align*}
Notice that,
\begin{align*}
  \int_V\nabla(\vec M\cdot\vec B)=\int_S\vec n(\vec M\cdot\vec B)da
\end{align*}
Two terms will cancel out. Here remains,
\begin{align*}
  \vec F=\int_V[(\vec M\cdot\nabla)\vec B+(\vec B\cdot\nabla)\vec M]d^3x-\int_S\vec M(\vec B\cdot\vec n)da
\end{align*}
Integrate by parts,
\begin{align*}
  \int_V(\vec M\cdot\nabla)\vec B=\int_S(\vec M\cdot\vec n)\vec B da-\int_V(\nabla\cdot M)\vec B d^3x
\end{align*}
\begin{align*}
  \int_V(\vec B\cdot\nabla)\vec M=\int_S(\vec B\cdot\vec n)\vec M da-\int_V(\nabla\cdot\vec B)\vec M d^3x
\end{align*}
Considering $\nabla\cdot\vec B=0$, in general,
\begin{align*}
  \vec F=-\int_V(\nabla\cdot\vec M)\vec B d^3x+\int_S(\vec M\cdot\nabla)\vec Bda
\end{align*}
\pagebreak

\section*{Jackson 5.20(b)}

The sphere has an uniform magnetization,
\begin{align*}
  \vec M=M_0\sin\theta_0\cos\phi_0\hat x+M_0\sin\theta_0\sin\phi_0\hat y+M_0\cos\theta_0\hat z
\end{align*}
In the external magnetic field,
\begin{align*}
  \vec B=B_0(1+\beta y)\hat x+B_0(1+\beta x)\hat y\\
  =B_0(1+\beta R\sin\theta\sin\phi)\hat x+B_0(1+\beta R\sin\theta\cos\phi)\hat y
\end{align*}
The normal unit vector will be
\begin{align*}
  \vec n=\sin\theta\cos\phi\hat x+\sin\theta\sin\phi\hat y+\cos\theta\hat z
\end{align*}
The force acting on the sphere,
\begin{align*}
  \vec F=-\int_V(\nabla\cdot\vec M)\vec B d^3x+\int_S(\vec M\cdot\vec n)\vec B da
\end{align*}
The first term vanishes since $\vec M$ is a constant.
\begin{align*}
  F_x&=M_0B_0\int_S(\sin\theta_0\sin\theta\cos\phi_0\cos\phi+\sin\theta_0\sin\theta\sin\phi_0\sin\phi+\cos\theta_0\cos\theta)(1+\beta R\sin\theta\sin\phi)da\\
  &=M_0B_0R^2\int_0^{2\pi}\int_{-1}^{1}(\sin\theta_0\sin\theta\cos(\phi-\phi_0)+\cos\theta_0\cos\theta)(1+\beta R\sin\theta\sin\phi)d\phi d\cos\theta\\
     &=\pi\beta M_0B_0R^3\sin\theta_0\sin\phi_0\int_{-1}^{1}\sin^2\theta d\cos\theta\\
  &=\frac{4}{3}\pi\beta M_0B_0R^3\sin\theta_0\sin\phi_0\\
  F_y&=M_0B_0\int_S(\sin\theta_0\sin\theta\cos\phi_0\cos\phi+\sin\theta_0\sin\theta\sin\phi_0\sin\phi+\cos\theta_0\cos\theta)(1+\beta R\sin\theta\cos\phi)da\\
     &=M_0B_0R^2\int_0^{2\pi}\int_{-1}^{1}(\sin\theta_0\sin\theta\cos(\phi-\phi_0)+\cos\theta_0\cos\theta)(1+\beta R\sin\theta\cos\phi)d\phi d\cos\theta\\
  &=\pi\beta M_0B_0R^3\sin\theta_0\cos\phi_0\int_{-1}^{1}\sin^2\theta d\cos\theta\\
     &=\frac{4}{3}\pi\beta M_0B_0R^3\sin\theta_0\cos\phi_0\\
  F_z&=0
\end{align*}
\pagebreak

\section*{Jackson 5.21(a)}

For a localized distribution of permanent magnetization, and in absence of macroscopic conduction currents, the magnetic field can be described by a scalar potential,
\begin{align*}
  \vec H=-\nabla\Phi_M
\end{align*}
The integral,
\begin{align*}
  \int\vec B\cdot\vec Hd^3x&=-\int\vec B\cdot\nabla\Phi_M\\
                           &=\int\Phi_M\nabla\cdot\vec B d^3x-\int_S\Phi_M\vec B\cdot d\vec a
\end{align*}
The first term vanishes since $\nabla\cdot\vec B=0$. The magnetization distribution is localized, so the second term also vanishes when integrate over all space. Hence,
\begin{align*}
  \int \vec B\cdot\vec Hd^3 x=0
\end{align*}
when the integral is taken over all space.
\pagebreak

\section*{Jackson 5.21(b)}

The potential energy for a differential element of the magnetization distribution,
\begin{align*}
  dW=-d\vec m\cdot\vec B
\end{align*}
Here $\vec B$ is caused by all the magnetization distribution but without the differential element $d\vec m$.The total magnetostatic energy can be written as,
\begin{align*}
  W=-\frac{1}{2}\int d\vec m\cdot\vec B=-\frac{1}{2}\int \vec M\cdot\vec B d^3x
\end{align*}
The factor $1/2$ yields from the fact that we double counted every element. Considering
\begin{align*}
  \vec B=\mu_0(\vec H+\vec M)
\end{align*}
\begin{align*}
  W=-\frac{\mu_0}{2}\int \vec M\cdot\vec H d^3x-\frac{\mu_0}{2}\int\vec M^2 d^3x
\end{align*}
The second term can be considered as an additive constant depends on the distribution of the magnetization.
\begin{align*}
  W=-\frac{\mu_0}{2}\int\vec M\cdot\vec H d^3x
\end{align*}
Now use,
\begin{align*}
  \vec M=\frac{\vec B}{\mu_0}-\vec H
\end{align*}
\begin{align*}
  W=-\frac{1}{2}\int (\vec B\cdot\vec H+\frac{\mu_0}{2}\vec H\cdot \vec H) d^3x
\end{align*}
The first term vanishes when integrate over all space, then
\begin{align*}
  W=\frac{\mu_0}{2}\int\vec H\cdot\vec H d^3x
\end{align*}
\pagebreak

\section*{Jackson 5.26}

% The self-inductance of the circuit,
% \begin{align*}
%   L&=\frac{\mu_0}{4\pi I^2}\int d^3x \int d^3x' \frac{\vec J(\vec x)\cdot\vec J(\vec x')}{|\vec x-\vec x'|}\\
%   &=\frac{1}{I^2}\int d^3x\vec J(\vec x)\cdot\vec A(\vec x)
% \end{align*}
% In this case, $\vec J$ is nonzero only when inside the parallel wires. The magnetic induction is azimuthal,
% \begin{align*}
%   \vec B=\frac{\mu_0 I}{2\pi a}\frac{\rho_<}{\rho_>}\hat\phi+\frac{\mu_0 I}{2\pi b}\frac{\rho'_<}{\rho'_>}\hat\phi'
% \end{align*}
% Considering,
% \begin{align*}
%   L&=\frac{1}{I^2}\int\frac{\vec B\cdot\vec B}{\mu_0}d^3x\\
%   &=\frac{\mu_0}{4\pi^2}\int (\frac{\rho_<}{a\rho_>}\hat\phi+\frac{\rho'_<}{b\rho'_>}\hat\phi')^2d^3x
% \end{align*}
% Here $\rho_<$($\rho_>$) is the smaller (larger) of $a$ and $\rho$. And $\rho'_<$($\rho'_>$) is the smaller (larger) of $b$ and $\rho'$. And in polar coordinates,
% \begin{align*}
%   \rho'=\sqrt{\rho^2+d^2-2d\rho\cos\phi}
% \end{align*}
% \begin{align*}
%   \hat\phi\cdot\hat\phi'=\cos(\phi+\phi')
% \end{align*}
% Try to evaluate the integral in three regions,
% \begin{align*}
%   L_1&=\frac{\mu_0}{4\pi^2}\int_0^a\int_0^{2\pi}(\frac{1}{a^2}\frac{\rho^2}{a^2}+\frac{2}{ab}\frac{\rho b}{a\rho'}\hat\phi\cdot\hat\phi'+\frac{1}{b^2}\frac{b^2}{\rho'^2})\rho d\rho d\phi\\
%      &=\frac{\mu_0}{8\pi}+
% \end{align*}
% \begin{align*}
%   L_2&=\frac{\mu_0}{4\pi^2}\int_a^d\int_0^{2\pi}(\frac{1}{\rho^2}+\frac{2}{\rho\rho'}\hat\phi\cdot\hat\phi'+\frac{1}{\rho'^2})\rho d\rho d\phi\\
%   &=\frac{\mu_0}{2\pi}\ln(\frac{d}{a})
% \end{align*}
% \begin{align*}
%   L_3=\frac{\mu_0}{4\pi^2}\int_0^{b}\int_0^{2\pi}(\frac{1}{\rho^2}+\frac{2}{b^2}\frac{\rho'}{\rho}\hat\phi\cdot\hat\phi'+\frac{\rho'}{b^4})\rho'd\rho'd\phi'
%   \end{align*}
 
If the two-wire transmission line is long enough, we can ignore the connecting parts. The seft-inductance of the circuit can be expressed as the total self-inductances from two infinite long wire with radius $a$, $b$.\par
The magnetic field induced by one wire,
\begin{align*}
  \vec B_a=\frac{\mu_0}{2\pi a}\frac{\rho_<}{\rho_>}\hat\phi
\end{align*}
The self-inductance,
\begin{align*}
  L_a&=\frac{1}{I^2}\int\frac{\vec B\cdot\vec B}{\mu_0}d^3x\\
     &=\frac{\mu_0}{4\pi^2}\int_0^{2\pi}d\phi[\int_0^a\frac{\rho^3}{a^4}d\phi+\int_a^d\frac{1}{\rho}d\phi]\\
     &=\frac{\mu_0}{8\pi}+\frac{\mu_0}{4\pi}\ln\frac{d}{a}
\end{align*}
In the similar way,
\begin{align*}
  \vec B_b=\frac{\mu_0}{2\pi b}\frac{\rho'_<}{\rho'_>}\hat\phi'
\end{align*}
\begin{align*}
  L_b&=\frac{1}{I^2}\int\frac{\vec B\cdot\vec B}{\mu_0}d^3x\\
  &=\frac{\mu_0}+\frac{\mu_0}{4\pi}\ln\frac{d}{b}
\end{align*}
The self-inductance of the circuit,
\begin{align*}
  L=L_a+L_b=\frac{\mu_0}{4\pi}+\frac{\mu_0}{2\pi}\ln\frac{d^2}{ab}
\end{align*}
\pagebreak

\section*{Jackson 5.27}
The magnetic induction,
\begin{align*}
B=\left\{\begin{array}{cc}
\frac{\mu_0I}{2\pi b}\frac{\rho_<}{\rho_>}\hat\phi & \rho<a\\
0 & \rho > a
\end{array}\right.
\end{align*}
Here $\rho_>=max\{b,\rho\}$ and $\rho_<=min\{b,\rho\}$. The self-inductance,
\begin{align*}
  L&=\frac{1}{I^2}\int\frac{\vec B\cdot\vec B}{\mu_0}d^3x\\
   &=\frac{\mu_0}{4\pi^2}\int_0^{2\pi}\int_0^b\frac{\rho^3}{b^4}d\rho d\phi+\frac{\mu_0}{4\pi^2}\int_b^a\int_0^b\frac{1}{\rho}d\rho d\phi\\
   &=\frac{\mu_0}{8\pi}+\frac{\mu_0}{4\pi}\ln(\frac{a^2}{b^2})
\end{align*}
When the inner conductor is a thin hollow tube,
\begin{align*}
  \vec B=\left\{
  \begin{array}{cc}
    0 & \rho<b\\
    \frac{\mu_0I}{2\pi \rho}\hat\phi & b<\rho<a\\
    0 & \rho>a
  \end{array}
  \right.
\end{align*}
The self-inductance,
\begin{align*}
  L&=\frac{1}{I^2}\int\frac{\vec B\cdot\vec B}{\mu_0}d^3x\\
   &=\frac{\mu_0}{4\pi^2}\int_0^{2\pi}\int_b^a\frac{1}{\rho}d\rho d\phi\\
   &=\frac{\mu_0}{4\pi}\ln(\frac{a^2}{b^2})
\end{align*}
% \bibliographystyle{abbrv}
% \bibliography{refs}
\end{document}