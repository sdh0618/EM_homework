\documentclass{article}
\usepackage[top=2cm,bottom=2cm]{geometry}
\usepackage{amsmath}
\usepackage{amsfonts}
\usepackage{latexsym}
\usepackage{graphicx}
\newcommand{\comment}[1]{}
\newcommand{\field}[1]{\mathbb{#1}} % requires amsfonts
\newcommand{\pd}[2]{\frac{\partial#1}{\partial#2}}
\begin{document}
\title{HW4 Classical Electromagnetism, Fall 2016}
\author{Deheng Song}
\maketitle
\section*{Jackson 3.1}

The general solution of the potential as a series in Legendre polynomials is
\begin{align*}
  \Phi(r,\theta)=\sum^\infty_{l=0}(A_l r^l+B_l r^{-(l+1)})P_l(\cos\theta)
\end{align*}
The boundary conditions on $r=a$ and $r=b$ gives
\[ V=\sum^\infty_{l=0}(A_l a^l+B_l a^{-(l+1)})P_l(\cos\theta),\quad 0<\cos\theta<1 \]
\[ V=\sum^\infty_{l=0}(A_l b^l+B_l b^{-(l+1)})P_l(\cos\theta),\quad -1<\cos\theta<0 \]
The coeffients are given by
\[ A_l a^l+ B_l a^{-(l+1)}= \frac{(2l+1)V}{2}\int_0^1P_l(x)dx \]
\[ A_l b^l +B_l b^{-(l+1)}=\frac{(2l+1)V}{2}(-)^l \int_0^1 P_l(x)dx \]
Evaluate the integral by Rodrigues's formula and solve $A_l$, $B_l$
\begin{align*}
   A_l =\frac{(2l+1)V}{2}\frac{b^{-(l+1)}-a^{-(l+1)}}{a^l b^{-(l+1)}- b^l a^{-(l+1)}}(-\frac 12)^{(l-1)/2}\frac{(l-2)!!}{2(\frac{l+1}{2})!} 
\end{align*}
\begin{align*}
  B_l=\frac{(2l+1)V}{2}\frac{b^l-a^l}{a^{-(l+1)}b^l-b^{-(l+1)}a^l}(-\frac 12)^{(l-1)/2}\frac{(l-2)!!}{2(\frac{l+1}{2})!}
\end{align*}
Write the potential to terms $l=4$, we have
\begin{align*}
  \Phi(r,\theta)=\frac{V}{2}+\frac{3V}{4}(\frac{b^{-2}-a^{-2}}{ab^{-2}-ba^{-2}}r+\frac{b-a}{a^{-2}b-b^{-2}a}r^{-2})\cos\theta\\-\frac{7V}{16}(\frac{b^{-4}-a^{-4}}{a^3b^{-4}-b^3a^{-4}}r^3+\frac{b^3-a^3}{a^{-4}b^{3}-b^{-4}a^3}r^{-4})(5\cos^3\theta-3\cos\theta)+...
\end{align*}
When $a\rightarrow 0$ and $b\rightarrow\infty$, the potential goes to
\[ \boxed{\Phi(r,\theta)=\frac{V}{2}+\frac{3V}{4}\frac{r}{a}\cos\theta-\frac{7V}{16}(\frac{r}{a})^3(5\cos^3\theta-3\cos\theta)+...}\]
\pagebreak

\section*{Jackson 3.5}
For $\vec x'$ is the vector from the orgin to the sphere, expand $1/|\vec x-\vec x'|$ in terms of spherical harmonics,
\[ \frac{1}{|\vec x-\vec x'|}=\frac{1}{\sqrt{r^2+a^2-2ar\cos\gamma}}=4\pi\sum^\infty_{l=0}\sum^{l}_{m=-l}\frac{1}{2l+1}\frac{r^l}{a^{l+1}}Y^*_{lm}(\theta',\phi')Y_{lm}(\theta,\phi) \]
Derivative both sides by r,
\begin{align*}
  \frac{r^2-ar\cos\gamma}{(r^2+a^2-2ar\cos\gamma)^{3/2}}=4\pi\sum^\infty_{l=0}\sum^{l}_{m=-l}\frac{l}{2l+1}\frac{r^l}{a^{l+1}}Y^*_{lm}(\theta',\phi')Y_{lm}(\theta,\phi)
\end{align*}
Derivative both sides by a,
\begin{align*}
  \frac{a^2-ar\cos\gamma}{(r^2+a^2-2ar\cos\gamma)^{3/2}}=4\pi\sum^\infty_{l=0}\sum^l_{m=-l}\frac{l+1}{2l+1}\frac{r^l}{a^{l+1}}Y^*_{lm}(\theta',\phi')Y_{lm}(\theta,\phi)
\end{align*}
Substract the both sides and multiply by $a$ and we find,
\[ \frac{a(a^2-r^2)}{4\pi}\frac{1}{(r^2+a^2-2ar\cos\gamma)^{3/2}}=\sum^\infty_{l=0}\sum^l_{m=-l}(\frac{r}{a})^lY^*_{lm}Y_{lm} \]
Thus,
\begin{align*}
  \frac{a(a^2-r^2)}{4\pi}\int\frac{V(\theta',\phi')}{(r^2+a^2-2ar\cos\gamma)^{3/2}}d\Omega'=\sum^\infty_{l=0}\sum^l_{m=-l}A_{lm}(\frac ra)^lY_{lm}(\theta,\phi)
\end{align*}
where
\[ A_{lm}=\int d\Omega'Y^*_{lm}(\theta',\phi')V(\theta',\phi') \]




\pagebreak
\section*{Jackson 3.7}

(a).Write the potential of the three point charges in free space,
\[ \Phi(\vec x)=\frac{1}{4\pi\varepsilon_0}(\frac{q}{|\vec x-a\hat z|}+\frac{q}{|\vec x+a\hat z|}-\frac{2q}{|\vec x|})\]
Since $1/|\vec x-\vec x'|$ can be expanded as
\[ \frac{1}{|\vec x-\vec x'|}=\sum^\infty_{l=0}\frac{r_<^l}{r_>^{l+1}}P_l(\cos\gamma) \]
Thus we can write the potential $\Phi$
\[ \boxed{\Phi(r,\theta)=\frac{q}{4\pi\varepsilon_0}[\sum^\infty_{l=1}\frac{r_<^l}{r_>^{l+1}}(P_l(\cos\theta)+P_l(-\cos\theta))]} \]
When $a\rightarrow 0$ and $q^2a=Q$ keeps finite, we have $r_<=a$ and $r_>=r$. Insert to the series, only the $l=2$  terms remain finite.
\[\boxed{\Phi(r,\theta)=\frac{Q}{2\pi\varepsilon_0}(r^{-3} P_2 (\cos\theta))} \]
where $P_2(\cos\theta)=\frac{1}{2}(3\cos^2\theta-1)$.\par

(b).To keep the potential vanishes on the shell, add two image charges outside the sphere and a constant.
\[ {\Phi(\vec x)=\frac{1}{4\pi\varepsilon_0}(\frac{q}{|\vec x-a\hat z|}+\frac{q}{|\vec x+a\hat z|}-\frac{bq}{a|\vec x-\frac{b^2}{a}\hat z|}-\frac{bq}{a|\vec x+\frac{b^2}{a}\hat z|}-\frac{2q}{|\vec x|}+\frac{2q}{b})} \]
The last term can be interpreted as the contribution of induced charges on the surface of the sphere.\\
Write the potential as the expansion of the Legendre polynomials,
\[
  \boxed{
  \Phi(r,\theta)=\frac{q}{4\pi\varepsilon_0}\sum^\infty_{l=0}\{ \frac{r_<^l}{r_>^{l+1}}[P_l(\cos\theta)+P_l(-\cos\theta)]+\frac{b}{a}\frac{r^l}{(b^2/a)^{l+1}}[P_l(\cos\theta)+P_l(-\cos\theta)]\}-\frac{1}{4\pi\varepsilon_0}(\frac{2q}{r}-\frac{2q}{b})
  }
\]
Again, when $a\rightarrow 0$ and $qa^2=Q$, keep the $l=2$ terms.
\[ \boxed{\Phi(r,\theta)=\frac{q}{2\pi\varepsilon_0}[r^{-3}-\frac{r^2}{b^5}]P_2(\cos\theta)} \]
\pagebreak

\section*{Jackson 3.14}

(a).The Green function with a spherical boundary at $r=b$ can be expanded as
\[ G(\vec x,\vec x')=4\pi\sum_{l,m}\frac{1}{2l+1}[\frac{r_<^l}{r_>^{l+1}}-\frac{(rr')^l}{b^{2l+1}}]Y^*_{lm}(\theta',\phi')Y_{lm}(\theta,\phi)   \]
The charge density can be writen as
\[\rho=\frac{A}{r^2}(d^2-r^2)[\delta(\cos\theta-1)+\delta(\cos\theta+1)],\qquad 0<r<d \]
where $r^2$ is to make sure above a linear charge density. Integral in spherical coordinates,
\[ \int_0^{2\pi}\int_0^{\pi}\int_0^d\frac{A}{r^2}(d^2-r^2)[\delta(\cos\theta-a)+\delta(\cos\theta+1)]r^2d\phi d\cos\theta dr=Q \]
Thus,
\[2\pi A\int_0^d (d^2-r^2)dr=Q\]
\[ A=\frac{3Q}{8\pi d^3}\]
For the boundary is grounded, we have $\Phi(\vec x')=0$ on the shell. Hence the potential inside the sphere is
\[\Phi(\vec x)=\frac{1}{4\pi\varepsilon_0}\int_V\rho(\vec x')G(\vec x,\vec x')d^3 x'\]
Substitute $\rho$ and $G(\vec x,\vec x')$, since the problem has azimuthal symmetry, only $m=0$ terms in Green fucntion remain.
\begin{align*}
  \Phi(\vec x)=\frac{3Q}{16\pi\varepsilon_0 d^3}\sum^\infty_{l=0}[P_l(1)+P_l(-1)]P_l(\cos\theta)\int_0^r(d^2-r'^2)r_<^l[\frac{1}{r_>^{l+1}}-\frac{r_>^l}{b^{2l+1}}]dr'
\end{align*}
For $r>d$, the integral can be evaluated,
\begin{align*}
  \Phi(\vec x)&=\frac{3Q}{16\pi\varepsilon_0 d^3}\sum^\infty_{l=0}[P_l(1)+P_l(-1)]P_l(\cos\theta)[\frac{1}{r^{l+1}}-\frac{r^l}{b^{2l+1}}]\int_0^d(d^2-r'^2)r'^ldr'\\
              &=\frac{3Q}{16\pi\varepsilon_0 d^3}\sum^\infty_{l=0}[P_l(1)+P_l(-1)]P_l(\cos\theta)[\frac{1}{r^{l+1}}-\frac{r^l}{b^{2l+1}}]\frac{2d^{l+3}}{(l+1)(l+3)}
\end{align*}
For $r<d$, divide the integral into two parts,
\begin{align*}
  \int_0^d&=[\frac{1}{r^{l+1}}-\frac{r^l}{b^{2l+1}}]\int_0^r(d^2-r'^2)r'^ldr'+r^{l}\int_r^d(d^2-r'^2)[\frac{1}{r'^{l+1}}-\frac{r'^l}{b^{2l+1}}]\\
  &=[\frac{1}{r^{l+1}}-\frac{r^l}{b^{2l+1}}](\frac{d^2r^{l+1}}{l+1}-\frac{r^{l+3}}{l+3})\\&+r^l[\frac{(l-2)d^2r^{-l}-lr^{2-l}}{l(l-2)}+\frac{(l+3)d^2r^{l+1}-(l+1)r^{l+3}}{b^{2l+1}(l+1)(l+3)}+\frac{2d^{2-l}}{l(l-2)}-\frac{2d^{l+3}}{b^{2l+1}(l+1)(l+3)}]
\end{align*}\par
(b).
The charge density on the surface is
\[\sigma=\varepsilon_0\pd{\Phi}{r}|_{r=b>d}=-\frac{3Q}{16\pi d^3}\sum^\infty_{l=0}[P_l(1)+P_l(-1)]P_l(\cos\theta)(2l+1)\frac{1}{b^{l+2}}\frac{2d^{l+3}}{(l+1)(l+3)}\]\par
\pagebreak
(c). When $d<<b$, we have $\frac{d^l}{r^l}$ and $\frac{d^l}{b^l}$ factors in the potential and charge density. Only the $l=0$ terms are important.\\
In this case,
\begin{align*}
  \Phi(x,\theta)&=\frac{3Q}{16\pi\varepsilon_0 d^3}[1+1][\frac 1r-\frac 1b]\frac{2d^3}{3}\\
                &=\frac{Q}{4\pi\varepsilon_0}[\frac 1r-\frac 1b]              
\end{align*}
And
\begin{align*}
  \sigma&=-\frac{3Q}{16\pi d^3}[1+1]\frac{1}{b^2}\frac{2d^3}{3}\\
        &=-\frac{Q}{4\pi}\frac{1}{b^2}
\end{align*}
The first term in the potential is the potential of point charge Q at free space, the second term is the contribution from the induced charge on the surface of the sphere.\\
The charge density describes the induced charge on the surface with total charge $-Q$.


% \bibliographystyle{abbrv}
% \bibliography{refs}
\end{document}